\documentclass[conference]{IEEEtran}

% Packages
\usepackage[utf8]{inputenc}
\usepackage[vietnam]{babel}
\usepackage{graphicx}
\usepackage{amsmath}
\usepackage{amssymb}
\usepackage{hyperref}
\usepackage{cite}
\usepackage{algorithm}
\usepackage{algorithmic}
\usepackage{booktabs}
\usepackage{multirow}
\usepackage{xcolor}

% Title
\title{Global Conflict Map: Hệ Thống Trực Quan Hóa Dữ Liệu Xung Đột Toàn Cầu Sử Dụng D3.js}

% Authors
\author{
    \IEEEauthorblockN{Tên Tác Giả 1}
    \IEEEauthorblockA{
        Khoa CNTT\\
        Trường Đại học ABC\\
        Email: author1@email.com
    }
    \and
    \IEEEauthorblockN{Tên Tác Giả 2}
    \IEEEauthorblockA{
        Khoa CNTT\\
        Trường Đại học ABC\\
        Email: author2@email.com
    }
}

\begin{document}

\maketitle

% Abstract
\begin{abstract}
Bài báo này trình bày hệ thống trực quan hóa dữ liệu xung đột toàn cầu (Global Conflict Map), một ứng dụng web tương tác được phát triển nhằm phân tích và minh họa các sự kiện xung đột bạo lực trên phạm vi thế giới. Hệ thống sử dụng bộ dữ liệu UCDP Georeferenced Event Dataset (GED) để cung cấp cái nhìn tổng quan về các xung đột vũ trang từ năm 1989 đến 2023. Được xây dựng trên nền tảng công nghệ D3.js và TopoJSON, ứng dụng cung cấp hai chế độ xem chính: Map View (chế độ bản đồ địa lý) và Graph View (chế độ đồ thị quan hệ), cho phép người dùng khám phá dữ liệu theo nhiều góc độ khác nhau. Kết quả cho thấy hệ thống hỗ trợ hiệu quả việc phân tích xu hướng xung đột, xác định các khu vực nóng, và tìm hiểu mối quan hệ giữa các phe phái.
\end{abstract}

% Keywords
\begin{IEEEkeywords}
Trực quan hóa dữ liệu, D3.js, xung đột toàn cầu, UCDP GED, bản đồ tương tác, phân tích dữ liệu
\end{IEEEkeywords}

% I. Introduction
\section{Giới thiệu}

Xung đột vũ trang là một trong những vấn đề nghiêm trọng nhất của nhân loại, gây ra hậu quả nặng nề về người và tài sản. Việc hiểu và phân tích các xu hướng xung đột đóng vai trò quan trọng trong công tác nghiên cứu hòa bình, hoạch định chính sách, và can thiệp nhân đạo \cite{ucdp2023}.

Bộ dữ liệu UCDP Georeferenced Event Dataset (GED) là một trong những nguồn dữ liệu toàn diện nhất về các sự kiện xung đột có vũ trang trên toàn thế giới, với hơn 250,000 sự kiện được ghi nhận từ năm 1989 đến nay. Tuy nhiên, việc phân tích trực tiếp bộ dữ liệu này đòi hỏi các công cụ trực quan hóa phù hợp.

Trong bài báo này, chúng tôi trình bày \textbf{Global Conflict Map}, một hệ thống web-based cho phép:
\begin{itemize}
    \item Trực quan hóa các sự kiện xung đột trên bản đồ thế giới
    \item Phân tích theo thời gian với thanh trượt năm tương tác
    \item Khám phá mối quan hệ giữa các quốc gia và phe phái
    \item Lọc và tìm kiếm theo khu vực, loại bạo lực
\end{itemize}

% II. Related Work
\section{Các nghiên cứu liên quan}

\subsection{Trực quan hóa dữ liệu địa lý}
Các hệ thống trực quan hóa dữ liệu địa lý như Leaflet \cite{leaflet}, Mapbox, và Google Maps đã được sử dụng rộng rãi. D3.js \cite{bostock2011d3} nổi bật với khả năng tùy biến cao và tích hợp trực tiếp với SVG.

\subsection{Phân tích dữ liệu xung đột}
Các nghiên cứu trước đây tập trung vào phân tích thống kê các xu hướng xung đột \cite{pettersson2021}. Tuy nhiên, ít có hệ thống cung cấp giao diện trực quan tương tác cho người dùng không chuyên.

% III. System Design
\section{Thiết kế hệ thống}

\subsection{Kiến trúc tổng quan}

Hệ thống được thiết kế theo mô hình client-side với các thành phần chính như Hình \ref{fig:architecture}:

\begin{figure}[htbp]
    \centering
    \fbox{\parbox{0.9\linewidth}{
        \centering
        \textbf{[Sơ đồ kiến trúc hệ thống]}\\
        \small Data Layer $\rightarrow$ Processing Layer $\rightarrow$ Visualization Layer $\rightarrow$ UI Layer
    }}
    \caption{Kiến trúc hệ thống Global Conflict Map}
    \label{fig:architecture}
\end{figure}

\begin{itemize}
    \item \textbf{Data Layer}: Đọc và parse dữ liệu CSV (GED Dataset)
    \item \textbf{Processing Layer}: DataManager, DataFilterManager xử lý indexing và caching
    \item \textbf{Visualization Layer}: RenderingEngine, ChartRenderer tạo các thành phần trực quan
    \item \textbf{UI Layer}: Giao diện tương tác 3 cột responsive
\end{itemize}

\subsection{Các module chính}

\subsubsection{DataFilterManager}
Module quản lý việc lọc và lập chỉ mục dữ liệu, cung cấp các phương thức:
\begin{itemize}
    \item \texttt{getEventsUpToYear(year)}: Lấy sự kiện đến năm chỉ định
    \item \texttt{getCountryEvents(countryName)}: Lọc theo quốc gia
    \item \texttt{getFactionEvents(faction)}: Lọc theo phe phái
\end{itemize}

\subsubsection{RenderingEngine}
Chịu trách nhiệm render các bubble (bọt) đại diện cho sự kiện xung đột trên bản đồ, với kích thước tỷ lệ thuận với số thương vong.

\subsubsection{ChartRenderer}
Tạo các biểu đồ thống kê bao gồm:
\begin{itemize}
    \item Timeline chart (biểu đồ dòng thời gian)
    \item Horizontal bar chart (biểu đồ thanh ngang)
    \item Heatmap (bản đồ nhiệt theo năm-tháng)
\end{itemize}

% IV. Implementation
\section{Hiện thực hệ thống}

\subsection{Công nghệ sử dụng}

\begin{table}[htbp]
    \centering
    \caption{Công nghệ và thư viện sử dụng}
    \label{tab:technologies}
    \begin{tabular}{@{}ll@{}}
        \toprule
        \textbf{Thành phần} & \textbf{Công nghệ} \\
        \midrule
        Trực quan hóa & D3.js v7 \\
        Bản đồ địa lý & TopoJSON \\
        Giao diện & HTML5, CSS3 \\
        Logic xử lý & Vanilla JavaScript \\
        Dữ liệu & UCDP GED v25.1 \\
        \bottomrule
    \end{tabular}
\end{table}

\subsection{Hai chế độ xem chính}

\subsubsection{Map View (Chế độ Bản đồ)}
Hiển thị bản đồ thế giới với các bubble đại diện cho sự kiện xung đột. Người dùng có thể:
\begin{itemize}
    \item Zoom vào quốc gia cụ thể bằng cách click
    \item Sử dụng thanh trượt thời gian để xem theo năm
    \item Lọc theo loại bạo lực và khu vực
\end{itemize}

\subsubsection{Graph View (Chế độ Đồ thị)}
Hiển thị mối quan hệ giữa các phe phái dưới dạng đồ thị force-directed. Cung cấp:
\begin{itemize}
    \item Visualization các phe phái như các node
    \item Quan hệ xung đột như các edge
    \item Click để xem chi tiết hoạt động của phe phái
\end{itemize}

\subsection{Bố cục giao diện}

Giao diện được thiết kế theo layout 3 cột (Hình \ref{fig:layout}):

\begin{figure}[htbp]
    \centering
    \fbox{\parbox{0.9\linewidth}{
        \centering
        \begin{tabular}{|c|c|c|}
            \hline
            Left Panel & Center & Right Panel \\
            (Filters) & (Map/Graph) & (Charts) \\
            \hline
        \end{tabular}
    }}
    \caption{Bố cục giao diện 3 cột}
    \label{fig:layout}
\end{figure}

\begin{itemize}
    \item \textbf{Left Panel}: Thống kê tổng quan, bộ lọc khu vực/loại bạo lực
    \item \textbf{Center}: Bản đồ hoặc đồ thị chính
    \item \textbf{Right Panel}: Biểu đồ chi tiết, danh sách sự kiện nghiêm trọng
\end{itemize}

% V. Features
\section{Các tính năng chính}

\subsection{Timeline Slider}
Thanh trượt thời gian cho phép người dùng xem dữ liệu theo từng năm từ 1989 đến 2023, với chức năng Play tự động phát.

\subsection{Interactive Bubbles}
Các bubble hiển thị sự kiện với:
\begin{itemize}
    \item Kích thước tỷ lệ với số thương vong (best estimate)
    \item Màu sắc phân biệt theo khu vực địa lý
    \item Hiệu ứng hover và click để xem chi tiết
\end{itemize}

\subsection{Faction Analysis}
Phân tích chi tiết các phe phái bao gồm:
\begin{itemize}
    \item Số sự kiện và tổng thương vong
    \item Các quốc gia hoạt động
    \item Biểu đồ hoạt động theo thời gian
    \item Activity Heatmap theo năm-tháng
\end{itemize}

\subsection{Event Details Panel}
Panel chi tiết sự kiện hiển thị:
\begin{itemize}
    \item Thông tin cơ bản (ngày, vị trí, loại bạo lực)
    \item Số liệu thương vong (low, best, high estimate)
    \item Các phe phái liên quan
    \item Nguồn tin và mô tả sự kiện
\end{itemize}

% VI. Dataset
\section{Bộ dữ liệu}

Hệ thống sử dụng UCDP GED v25.1 với các thuộc tính chính:

\begin{table}[htbp]
    \centering
    \caption{Các trường dữ liệu chính}
    \label{tab:dataset}
    \begin{tabular}{@{}lp{4.5cm}@{}}
        \toprule
        \textbf{Trường} & \textbf{Mô tả} \\
        \midrule
        id & Mã định danh sự kiện \\
        year & Năm xảy ra \\
        country & Quốc gia \\
        region & Khu vực địa lý \\
        latitude, longitude & Tọa độ \\
        best, low, high & Ước tính thương vong \\
        type\_of\_violence & Loại bạo lực (1-3) \\
        side\_a, side\_b & Các bên tham chiến \\
        \bottomrule
    \end{tabular}
\end{table}

Tổng số: \textbf{253,000+} sự kiện từ \textbf{130+} quốc gia.

% VII. Results
\section{Kết quả và đánh giá}

\subsection{Hiệu năng}
Hệ thống đạt được:
\begin{itemize}
    \item Thời gian tải dữ liệu: < 3 giây (file 250MB)
    \item Render mượt mà với 10,000+ bubble/frame
    \item Responsive trên các thiết bị khác nhau
\end{itemize}

\subsection{Tính năng phân tích}
Người dùng có thể:
\begin{itemize}
    \item Xác định xu hướng gia tăng/giảm xung đột theo thời gian
    \item Phát hiện các hotspot (điểm nóng) xung đột
    \item So sánh mức độ xung đột giữa các khu vực
    \item Truy vết hoạt động của các phe phái cụ thể
\end{itemize}

% VIII. Conclusion
\section{Kết luận}

Bài báo đã trình bày hệ thống Global Conflict Map - một công cụ trực quan hóa dữ liệu xung đột toàn cầu hiệu quả. Hệ thống cung cấp giao diện trực quan, tương tác, giúp người dùng dễ dàng khám phá và phân tích dữ liệu UCDP GED.

\subsection{Hướng phát triển}
Các hướng mở rộng trong tương lai:
\begin{itemize}
    \item Tích hợp Machine Learning để dự đoán xu hướng
    \item Thêm chế độ realtime với dữ liệu cập nhật
    \item Hỗ trợ export báo cáo và chia sẻ analysis
    \item Tích hợp thêm các nguồn dữ liệu khác (ACLED, GTD)
\end{itemize}

% Acknowledgment
\section*{Lời cảm ơn}
Nhóm tác giả xin chân thành cảm ơn Uppsala Conflict Data Program (UCDP) đã cung cấp bộ dữ liệu GED công khai phục vụ nghiên cứu.

% References
\begin{thebibliography}{9}

\bibitem{ucdp2023}
Uppsala Conflict Data Program (UCDP),
``UCDP Georeferenced Event Dataset (GED) Global version 25.1,''
2023. [Online]. Available: https://ucdp.uu.se/downloads/

\bibitem{bostock2011d3}
M. Bostock, V. Ogievetsky, and J. Heer,
``D3: Data-Driven Documents,''
\textit{IEEE Trans. Visualization and Computer Graphics}, vol. 17, no. 12, pp. 2301--2309, 2011.

\bibitem{leaflet}
V. Agafonkin,
``Leaflet - an open-source JavaScript library for interactive maps,''
2023. [Online]. Available: https://leafletjs.com/

\bibitem{pettersson2021}
T. Pettersson and M. Öberg,
``Organized violence, 1989--2020,''
\textit{Journal of Peace Research}, vol. 58, no. 4, pp. 809--825, 2021.

\end{thebibliography}

\end{document}
